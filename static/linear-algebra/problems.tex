%------------------------------------------------------------------------Begin preamble------------------------------------------------------------------------
\documentclass[12pt]{article}
% Uncomment these two lines if you want to use Times New Roman (needs XeLaTeX)
% \usepackage{fontspec}
% \setmainfont{Times New Roman}

\usepackage{amsmath} % For math
\usepackage{amssymb} % For more math
\usepackage{fancyhdr} % For fancy headers and footers
\usepackage{fancyvrb} % For writing blocks of code verbatim (like LaTeX code)
\usepackage{geometry} % For manipulating margins and meta doc stuff
\usepackage{graphicx} % Required for inserting images
\usepackage{hyperref} % For linking stuff
\usepackage{cleveref} % Better cross refencing % Must be loaded after hyperref
\usepackage{lastpage} % For doing "Page n of |n|"
\usepackage{minted} % For color coding code sections
\usepackage{pgfplots} % For adding plots and graphs
\usepackage[indent=00pt]{parskip} % Get rid of beginning indents on paragraphs and set spacing between paragraphs
\usepackage{siunitx} % For adding siunits inside math zones
\usepackage[most]{tcolorbox} % For inserting color boxes
\usepackage{tikz}
\usepackage{titling} % For vertically centering title, author, and date
\usepackage{tocloft} % For cooler toc
\usepackage{xcolor} % For coloring text
\usepackage{witharrows} % For cool arrows for pointing at stuff

%%%%%%%%% Begin tocloft setup %%%%%%%%%%%%%%
\renewcommand\cftsecfont{\normalfont}
\renewcommand\cftsecpagefont{\normalfont}
\renewcommand{\cftsecleader}{\cftdotfill{\cftsecdotsep}}
\renewcommand\cftsecdotsep{\cftdot}
\renewcommand\cftsubsecdotsep{\cftdot}
%%%%%%%%% End tocloft %%%%%%%%%%%%%%

%%%%%%%%%% Begin hyperref setup %%%%%%%%%%%%%%%%%%%
\hypersetup{
    colorlinks=true,
    linkcolor=blue,
    urlcolor=blue,
}
%%%%%%%%%% End hyperref setup %%%%%%%%%%%%%%%%%%%

%%%%%%% Begin pgfplotsset setup %%%%%%%
\pgfplotsset{compat=1.18} % Compiler be bugging
%%%%%%% End pgfplotsset setup %%%%%%%

%%%%%%% Begin fancyhdr setup %%%%%%%
\pagestyle{fancy}
\renewcommand{\footrulewidth}{0.4pt} % default is 0pt
\lhead{Linear Algebra} % Top left header
\chead{} % Top center header
\rhead{} % Top right header

\lfoot{Brandon J. T. Noguera} % Bottom left footer
\cfoot{} % Bottom center footer
\rfoot{Page \thepage\ of \pageref*{LastPage}} % Bottom right footer
% doing \pageref*{some_url} turns off the hyperlink capability for some_url
%%%%%%% End fancyhdr setup %%%%%%%

%%%%%%% Begin titlingpage setup %%%%%%%
\renewcommand\maketitlehooka{\null\mbox{}\vfill}
\renewcommand\maketitlehookd{\vfill\null}
%%%%%% End titlingpage setup %%%%%%%
%------------------------------------------------------------------------End preamble------------------------------------------------------------------------

\title{\textbf{Linear Algebra}}
\author{Brandon Jose Tenorio Noguera}
\begin{document}

\begin{titlingpage}
\maketitle
\end{titlingpage}

\newpage
\tableofcontents
\newpage

\section{Exercises on the geometry of linear equation}\label{p1}
\begin{tcolorbox}[title=Source]
\href{https://ocw.mit.edu/courses/18-06sc-linear-algebra-fall-2011/resources/mit18_06scf11_ses1-1prob/}{Source}
\end{tcolorbox}

\subsection{Problem 1.1}\label{p1.1}
\begin{tcolorbox}[title=Problem 1.1]
1.3 \#4. \emph{Introduction to Linear Algebra}: Strang. Find a combination $ x_1 \vec{w_1} + x_2 \vec{w_2} $ that gives the zero vector
  \begin{equation*}
    \vec{w_1} = \begin{bmatrix}
    1\\
    2\\
    3
    \end{bmatrix}
    \vec{w_2} = \begin{bmatrix}
    4\\
    5\\
    6
    \end{bmatrix}
    \vec{w_3} = \begin{bmatrix}
    7\\
    8\\
    9
    \end{bmatrix}
  \end{equation*}
  Those vectors are (independent/dependent)
  The three vectors lie in a [BLANK] 
  The matrix $ W $ with those columns is not invertible
\end{tcolorbox}

We must find scalars $ x_1, x_2, x_3 $ such that
\begin{align*}
  x_1 \vec{w_1} + x_2 \vec{w_2} + x_3 \vec{w_3} &= \vec{0}\\ 
  \intertext{or written differently...}\\
  x_1 \begin{bmatrix}
  1\\
  2\\
  3
  \end{bmatrix} +
  x_2 \begin{bmatrix}
  4\\
  5\\
  6
  \end{bmatrix} + 
  x_3 \begin{bmatrix}
  7\\
  8\\
  9
\end{bmatrix} &=
  \begin{bmatrix}
  0\\
  0\\
  0
  \end{bmatrix}
\end{align*}

The obvious (and trivial) solution is to let every scalar be equal to 0
\begin{align*}
  x_1=0\\
  x_2=0\\
  x_3=0
\end{align*}

Another solution could be
\begin{align*}
  x_1 &= 1\\
  x_2 &= -2\\
  x_3 &= 1
\end{align*}

which would give us
\begin{align*}
  1(1) + -2(4) + 1(7) &= 0\\
  1(2) + -2(5) + 1(8) &= 0\\
  1(3) + -2(6) + 1(9) &= 0
\end{align*}

These vectors are \textbf{dependent} because there is a combination of the vectors that gives us the 0 vector.
Because the vectors are dependent, they must lie on the same plane




\subsection{Problem 1.2}\label{p1.2}
\begin{tcolorbox}[title=Problem 1.2]
Multiply: \begin{equation*}
  \begin{bmatrix}
    1 & 2 & 0\\
    2 & 0 & 3\\
    4 & 1 & 1
  \end{bmatrix}
  \begin{bmatrix}
  3\\
  -2\\
  1
  \end{bmatrix}
\end{equation*}
\end{tcolorbox}

\begin{align*}
  \begin{bmatrix}
    1 & 2 & 0\\ 
    2 & 0 & 3\\
    4 & 1 & 1 
  \end{bmatrix}
  \begin{bmatrix}
  3\\
  -2\\
  1
  \end{bmatrix} =
  3 \begin{bmatrix}
  1\\
  2\\
  4
  \end{bmatrix} +
  -2 \begin{bmatrix}
  2\\
  0\\
  1
  \end{bmatrix} + 
  1 \begin{bmatrix}
  0\\
  3\\
  1
  \end{bmatrix} =
  \begin{bmatrix}
  3(1) + -2(2) + 1(0)\\
  3(2) + -2(0) + 1(3)\\
  3(4) + -2(1) + 1(1)
  \end{bmatrix} =
  \begin{bmatrix}
  -1\\
  9\\
  11
  \end{bmatrix}
\end{align*}

\subsection{Problem 1.3}\label{p1.3}
\begin{tcolorbox}[title=Problem 1.3]
True or False. If matrix $ A = 3 \times 2 $, and matrix $ B = 2 \times 3 $, their product will equal a $ 3 \times 3 $ matrix $ AB $. If this is false, write a similar sentence which is correct
\end{tcolorbox}
This is \textbf{true}. You can only multiply two matrices if the columns of the first one equal the rows of the second one. Their product will be a matrix with the same number of rows as the first matrix, and the same number of columns as the second matrix

\begin{align*}
    A (m\ by\ n) \cdot B (n\ by\ p) = AB (m\ by\ p)
\end{align*}

\end{document}
